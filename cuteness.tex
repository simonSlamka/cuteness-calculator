\documentclass{article}
\usepackage{csquotes}
\usepackage[english]{babel}
\usepackage{amsmath}
\usepackage{graphicx}
\usepackage{hyperref}
\usepackage{multirow}
\usepackage{microtype}
\usepackage[backend=biber, sorting=none]{biblatex}
\usepackage{listings}
\lstset{breaklines=true,basicstyle=\footnotesize,language=Python}
\usepackage[normalem]{ulem}
\addbibresource{refs.bib}
\AtBeginBibliography{\raggedright}
%\usepackage{multicol}
\usepackage{emoji}

\title{Female cuteness and the `kawaii' aesthetic}
\author{Simon Slamka\\
\small\textbf{OngakkenAI}}
\date{2023}

\begin{document}

\maketitle

\begin{abstract}

\end{abstract}

\section{Introduction}
\subsection{Key terms}
Cuteness, Simtoonian Cuteness Index, kawaii, beauty, female, woman, girl, machine learning, image classification, computer vision, deep learning, convolutional neural networks, Transformers, dating, cosmetics, entertainment

\subsection{Definitions}
\begin{itemize}
    \item \textbf{Cuteness} - the degree to which a female subject elicits a positive, non-sexual, behavioral response in an observer, as determined through subjective ratings of female portraits by observers
    \item \textbf{Simtoonian Cuteness Index} - a normalized composite score of relative cuteness %^ TODO: relative/absolute??!
    \item \textbf{Kawaii} - the Japanese word for cute, but with a more specific meaning
    \item \textbf{Artificial intelligence} - the field of computer science that deals with the simulation of intelligent behavior in computers
    \item \textbf{Computer vision} - a subfield of artificial intelligence that deals with the automatic extraction, analysis, and understanding of useful information from a single image or a sequence of images
    \item \textbf{Model} - a mathematical function that is used to make predictions about data
    \item \textbf{Neural network} - a type of machine learning model that is inspired by the structure of the brain
    \item \textbf{Deep learning} - a subfield of machine learning that uses neural networks with two or more hidden layers to learn representations of data with multiple levels of abstraction
    \item \textbf{Convolution} - a mathematical operation that takes two functions $f$ and $g$ and produces a third function that expresses how the shape of one is modified by the other
    \item \textbf{Convolutional layer} - a layer in a convolutional neural network that performs a convolution operation
    \item \textbf{Convolutional neural network} - a type of neural network that uses convolutional layers to extract features from images
    \item \textbf{Attention} - a mechanism that allows neural networks to focus on very specific parts of an input
    \item \textbf{Self-attention} - a type of attention mechanism that allows neural networks to learn representations of data by relating different positions of a single sequence
    \item \textbf{the Transformer} - a type of neural network that uses self-attention to learn representations of data, originally developed by Vaswani et al.\cite{vaswani2017attention}
    \item \textbf{ViT} - Vision Transformer - a type of Transformer that uses self-attention to learn representations of images data\cite{wu2020visual}
    \item \textbf{Class} - a category of objects that a machine learning model can predict
    \item \textbf{Image classification} - a machine learning task that involves predicting the class of an image
    \item \emoji{hugging-face} - Hugging Face - a company that develops open-source libraries for natural language processing and computer vision
    \item \textbf{Layer} - a component of a neural network that performs a specific operation
    \item \textbf{Parameter} - a variable that is used in a neural network
    \item \textbf{Activation function} - a function that determines the output of a neural network
    \item \textbf{Loss function} - a function that measures how well a neural network performs a task
    \item \textbf{Optimizer} - a function that updates the parameters of a neural network
    \item \textbf{FFNN} - feed-forward neural network - a type of neural network that consists of an input layer, an output layer, and one or more hidden layers
    \item \textbf{ReLU} - rectified linear unit - a type of activation function that is used in neural networks
    \item \textbf{Softmax} - a type of activation function that is used in neural networks
    \item \textbf{Negative logarithmic likelihood} - cross-entropy loss - a type of loss function used for classification tasks
    \item \textbf{Adam} - Adaptive Moment Estimation - an optimizer that is used in modern neural networks
\end{itemize}

\subsection{Rationale}
% !!! PLACEHOLDER !!!
There is currently no software solution for determining and quantifying female cuteness, as per the definition provided in this paper. While this may seem irrelevant at first glance, such a solution could have significant applications in various fields such as entertainment, dating (for instance, in matchmaking or helping individuals identify preferred features during mate selection), cosmetics, psychology, and potentially, elsewhere.

Additionally, formalizing the concept of cuteness presents a unique challenge due to its inherently subjective nature. However, it is possible to identify common elements among opinions of cuteness, such as eye size and fluctuating asymmetry. These elements can serve as a starting point for developing a quantifiable measure of cuteness.

\subsection{Hypotheses}
\subsubsection{$H_1$}
\textit{Female cuteness can be quantified using a machine learning model with a predictive accuracy of at least 90\%.}

\subsubsection{$H_2$}
\textit{A custom, machine learning-based composite score of relative cuteness can be developed and used to quantify female cuteness.}

\subsubsection{$H_0$}
\textit{Female cuteness is not quantifiable using a machine learning model and as such, a custom, machine learning-based composite score of relative cuteness cannot be developed.}

\subsection{Premise}
% !!! PLACEHOLDER !!!
The perception of cuteness is a subjective and deeply personal experience, influenced by a myriad of factors ranging from cultural norms to individual preferences. However, certain physical attributes, such as facial symmetry and eye size, have been consistently associated with the perception of cuteness across different cultures and societies. This study assumes that these universally recognized attributes can be quantified and used as input features for a machine learning model. The model, trained on a diverse dataset of female portraits rated for cuteness, is expected to learn the complex, non-linear relationships between these features and the perceived cuteness. The output of this model, termed as the Simtoonian Cuteness Index, is proposed as a quantifiable measure of female cuteness. This premise is based on the success of machine learning models in other subjective domains such as sentiment analysis and aesthetic evaluation. It also assumes the availability of a sufficiently large and diverse dataset for training the model, and the feasibility of extracting the required features from the input images.

\subsection{Scope}
% !!! PLACEHOLDER !!!
This study is confined to the domain of aesthetic analysis. It does not involve any neurological, psychological, or behavioral research. The scope is strictly limited to the quantification of cuteness based on physical attributes and does not extend to the interpretation of these attributes in terms of underlying psychological or neurological processes. The study also does not make any claims about the behavioral implications of the perceived cuteness.

\subsection{Limitations}
% !!! PLACEHOLDER !!!
This study has several limitations that should be considered:

\begin{itemize}
    \item The machine learning model used in this study is trained on a specific dataset. The model's performance may vary when applied to different datasets.
    \item The study assumes that universally recognized attributes of cuteness can be quantified and used as input features for a machine learning model. However, the perception of cuteness is subjective and can vary greatly among individuals.
    \item The study does not consider the influence of cultural, societal, or personal factors on the perception of cuteness.
    \item The study is limited to the quantification of cuteness based on physical attributes and does not consider other aspects such as personality or behavior.
    \item The study assumes the availability of a sufficiently large and diverse dataset for training the model. The lack of such a dataset could limit the model's ability to generalize its predictions.
\end{itemize}

\subsection{Literature review}
\subsubsection{Related work}
\begin{itemize}
    \item 
\end{itemize}




\printbibliography
%\end{multicols}
\end{document}